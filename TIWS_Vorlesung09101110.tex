Terme: Die einzige Datenstruktur in Prolog
Zusätzlich zu Konstanten und Variablen brauchen wir Funktoren (gleiche Syntax wie Relationen und Konstanten).

Bsp. für Terme \mathbb{N} in symbolischer Notation.
5 \rightarrow s(s(s(s(s(o))))) \rightarrow $s^$5(o)
1-Stelliger Funktor \nearrow \uparrow Konstante

Terme-Definition.:										Bsp.: Listen
Induktionsanfang: - Jede Konstante c ist ein Term		nil
				  - Jede Variable X ist ein Term		2-stelliger Funktor

n \rightarrow n+1: - Für jeden n-stelligen Funktor f 	list
und n Terme t1,...,tn ist f(t1,...,tn) ein Term.

Term \rightarrow Baum
Bsp. Konstanten a, b Variablen X Funktoren $h^$(3), $g^$(1), $f^$(2)
$h^$(3)($g^$(1)($g^$(1)(a),$f^$(2)(X,b),$f^$(2)($f^$(2)(a,$g^$(1)(X)),b))