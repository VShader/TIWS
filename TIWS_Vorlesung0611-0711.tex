Neben Arithmetik spezielle Notation für Listen!

list(1,list(2,list(3,nil)))
								nicht mischen!!!
[1|[2|[3|[]]]]

Quicksort: Teile und Hersche / Divide and Conquer
Anwendung: Generate & Test-Verfahren

alle Permutationen			ist sortiert
der Eingabe erzeugen

psort([3,2,1],Zs) :- permutation([3,2,1]), ordered([3,2,1]) f
					 permutation([3,1,2]), ordered([3,1,2]) f
					 permutation([2,1,3]), ordered([2,1,3]) f
					 permutation([2,3,1]), ordered([2,3,1]) f
					 permutation([1,2,3]), ordered([1,2,3]) 
					 permutation([1,3,2]), ordered([1,3,2]) f
					 
ordered([]).
ordered([X]).
ordered([X,Y|Xs]) :- X<Y, ordered([Y|Xs]).

Schöne Anwendung: n-Damen-Problem!



Kowalski-Idee: Programming = Logic + Control
			   In Prolog verwirklicht?
			   D.h. nur Logik angeben, keine Gedanken zur Auswertung
			   D.h. sind "und" und "oder" kommutativ?

Fakten- und Regelreihenfolge

"oder" kommutativ?

Bsp.:
%istdrin(X,Xs): X ist in Liste Xs enthalten
istdrin(X,[X|Xs]).
istdrin(X,[_|Xs]) :- istdrin(X,Xs).



%istdrin(X,Xs): X ist in Liste Xs enthalten
istdrin(X,[_|Xs]) :- istdrin(X,Xs).
istdrin(X,[X|Xs]).			   