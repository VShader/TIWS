Übung 1 
Aufgabe 1
a) Eine Ordnung les besteht, wenn alle Wörter  mit ihrem Anfangsbuchstaben in alph Ordnung sind.
   Sollten mehrere Wörter gleich anfange, so nehme man den nächsten Buchstaben und sortiere daran 
   die Wörter.
b) Eine Ordnung geo ist dann gegeben wenn alle Elemente der Menge Millionenstädte von Ost nach 
   West sortiert sind.
 
  
Aufgabe 2
a) A * \lnot A
Kein Modell
A	A * \lnot A
0	0
1	0

b) (A+B)*(\lnot A * \lnot B)
Kein Modell
A	B	A+B	\lnot A * \lnot B	(A+B)*(\lnot A * \lnot B)
0	0	0	1	0
0	1	1	0	0
1	0	1	0	0
1	1	1	0	0

c) (A+B)*(\lnot A + B)
2 Modelle
A	B	A+B	\lnot A + B	(A+B)*(\lnot A + B)
0	0	0	1	0
0	1	1	1	1
1	0	1	0	0
1	1	1	1	1

d) (A+B)*(\lnot A + \lnot B)
2 Modelle
A	B	A+B	\lnot A + \lnot B	(A+B)*(\lnot A + \lnot B)
0	0	0	1	0
0	1	1	1	1
1	0	1	1	1
1	1	1	0	0


Aufgabe 3
a) -> kann als
A	B	\lnot A * B	\lnot (\lnot A * B)
0	0	0	1
0	1	1	0
1	0	0	1
1	1	0	1

b) XOR kann als (A+B)*(\lnot A + \lnot B) dargestellt werden.
A	B	A+B	\lnot A + \lnot B	(A+B)*(\lnot A + \lnot B)
0	0	0	1	0
0	1	1	1	1
1	0	1	1	1
1	1	1	0	0

Aufgabe 4
a) plus(X,Y,Z)
	plus(0,0,0), plus(0,1,1), plus(1,0,1), plus(1,1,2), plus(1,2,3) ...
	
b) plus(o,o,o), plus(o,s(o),s(o)), plus(s(o),o,s(o)), plus(s(o),s(o),s(s(o))), plus(s(o),s(s(o)),s(s(s(o)))) ....

c)Um die Relation aus a) vollständig zu beschreiben, müssen wir n-Elemente Angeben.
  Es müssen n Prolog-Fakten angegeben werden um die Relation aus a) in Prolog darzustellen.