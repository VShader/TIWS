Aufgabe 5:

mutter(hanna, silke).
mutter(silke, anna).
mutter(anna, sabrina).

Aufgabe 6:

a)

% add(X,Y, E) :- E = X + Y
add(X,o,X).
add(X,s(Y),E):- add(s(X),Y,E).

b)

add(Z, s(o), s(s(o))). // Es gibt ein Z für das gilt Z + 1 = 2 Umformung: Z = 2 - 1

c)

add(Z,Z,s(s(o))).
add(Z,Z,s(s(s(s(o))))). //Es gibt ein Z für das gilt Z + Z = 4 Umformung: 2Z = 4 |:2 -> Z = 2

d)

Im moment könnte unser Programm auch andere Darstellung verwenden, da wir dies noch nicht explizit unterbinden.
Wir führen die natürlichen Zahlen in symbolischer Darstellung ein mit folgender Regel:

natsymb(o).
natsymb(s(X)) :- natsymb(X).

neuer Fakt:

add(X,o,X) :- natsymb(X).

e)

Aufgabe 7:

a)

f

b)

c)



Aufgabe 8:

a)

eqZero(o).

b)

neqZero(s(o)).
neqZero(s(X)) :-  neqZero(X).

c)

less(o,X) :- neqZero(X).
less(s(X),s(Y)) :- less(X,Y).

d)

square(X,Y) :- mult(X,X,Y).

Aufgabe 10:

a)


b)

invList(nil). :- natsymb(X).
invList(list(X,Y)) :- natsymb(X),invList(Y).