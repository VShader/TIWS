Übung: 7

Aufgabe: 33
a) app mit Variablen im 1. & 3. Argument \infty Rechnung
Bei vollständig instantiiertem 1. Argument wird dieses bei Regelanwendung um 1 verkürzt \Rightarrow irgendwann ist die Liste im 1. Argument leer \Rightarrow Fakt wird angewendet.
\Rightarrow Hs ist vor Aufruf von app vollständig instantiiert.
\Rightarrow Stopp!

b) Ja, da in der Querry alles vollinstantiiert, liefert der Aufruf des 1. apps einen endlichen Teilbaum, da das 3. Argument vollinstantiiert ist.
Das 2. app liefert ebenfalls einen endlichen Teilbaum, da das 3. Argument durch das 1. app vollinstantiiert ist.

c) \infty , da nur 2. Argument im 1. app vollinstantiiert!


Aufgabe: 34
members(X,[X|Ys]).
members(X,[Y|Ys]) :- members(X,Ys).

add_el(X,Ls,Rs) :- members(X,Ls), !, app([],Ls,Rs).
add_el(X,Ls,Rs) :- app([X],Ls,Rs).

besser :

add_el(X,Ls,Ls) :- members(X,Ls), !.
add_el(X,Ls,[X|Ls]).


Aufgabe: 35
a) Steht der Cut am Anfang der rechten Regelseite, werden nur Alternativen von p() 
	weggeschnitten. \Rightarrow es kann mehrere Lösungen geben.
b) Steht der Cut am Ende der rechten Regelseite, werden alle Alternativen von p,g1,..,gn abgeschnitten.

	
Aufgabe: 36
Grüner Cut siehe Aufgabe 34.
ist deterministisch \Rightarrow es kann nur 1 Lösung geben.
alternativ: man kommt zu !, wenn member() = true \Rightarrow dann gilt 2. app nicht, d.h. man kann nur aus einer Regel Lösung bekommen.

Roter Cut add_el(X,Ls,Rs) :- !, member(X,Ls), app([],Ls,Rs).
		  add_el(X,Ls,Rs) :- app([X],Ls,Rs).

		  
Aufgabe: 37
a) \sum römisch ={I,V,X,L,C}
b) CCCXCIX = 399
c) Wörter={I,II,III,IV,V}
d) keineWörter={IIII,IIV,VIIII,IIX,XIIII}
e) CXLIV = 144


Aufgabe: 38
a) Informatik, Computer, Rechner
b) compare, Byrne, Buhter
c) Ich esse eine Byrne.
d) Ich esse Birne.
e) Die Informatik ist eine Wissenschaft für sich.
f) Nachts ist es kälter als draußen.