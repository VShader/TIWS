Übung: 7

Aufgabe: 34
members(X,[X|Ys]).
members(X,[Y|Ys]) :- members(X,Ys).

add_el(X,Ls,Rs) :- members(X,Ls), !, app([],Ls,Rs).
add_el(X,Ls,Rs) :- app([X],Ls,Rs).


Aufgabe: 35
a) Steht der Cut am Anfang der rechten Regelseite, werden alle Alternativen zu q1() bis qm() 
	weggeschnitten.
b) Steht der Cut am Ende der rechten Regelseite, kann er nur dann das Backtracken verhindern, 
	wenn die vorherigen Bedingungen erfolgreich waren.

	
Aufgabe: 36
Grüner Cut siehe Aufgabe 34.
Roter Cut add_el(X,Ls,Rs) :- !, member(X,Ls), app([],Ls,Rs).
		  add_el(X,Ls,Rs) :- app([X],Ls,Rs).

		  
Aufgabe: 37
a) \sum römisch ={I,V,X,L,C}
b) CCCXCIX = 399
c) Wörter={I,II,III,IV,V}
d) keineWörter={IIII,IIV,VIIII,IIX,XIIII}
e) CXLIV = 144


Aufgabe: 38
a) Informatik, Computer, Rechner
b) compare, Byrne, Buhter
c) Ich esse eine Byrne.
d) Ich esse Birne.
e) Die Informatik ist eine Wissenschaft für sich.
f) Nachts ist es kälter als draußen.