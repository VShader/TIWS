Übung 5:

20)
vater(abraham, isaak).
vater(gott, X).
?- vater(Z1, Z2).

						vater(Z1,Z2)
						/         \
sub1[Z1/abraham, Z2/isaak]		sub1[Z1/gott, Z2/ZH1, ZH1/X]
?-								?-


21)
ERROR: Out of local stack
Exception: (493,205) sohn(isaak, abraham) ? 


22)
a)
?- f(f(X,f(a,g(X))),g(f(b,Y))) = f(f(g(g(Z1)),Z2),g(f(Z3,f(Z3,a)))).
X = g(g(_G571))
Y = f(b, a)
Z1 = _G571
Z2 = f(a, g(g(g(_G571))))
Z3 = b ;

b)
f(f(X,f(a,g(X))),g(f(b,X))) = f(f(g(g(Z1)),Z2),g(f(Z3,f(Z3,a)))).
No

c)
Geht nicht weil, X nicht gleichzeitig g(X) sein kann. (kein Occur Check)


23)
app(Z1,Z2,Z2).

app(nil,Xs,Xs).
[Z1/nil,Z2/Xs].

app(list(X,X1s),Ys,list(X,X2s)).
[Z1/list(X,X1s),Z2/Ys,Ys/list(X,X2s)].


24)

						vater(Z1,Z2).
						/
sub1[Z1/abbraham,Z2/issak]
?-


25)
Backtracking versucht aus Teillösungen eine Gesamtlösung zu bilden. Führt eine Teillösung nicht zum Endgültigen Ergebnis wird 
der letzte Schritt zurückgenommen und stattdesen alternative Lösungswege ausprobiert, somit ist sichergestellt, dass alle in Frage 
kommenden Lösungswege ausprobiert werden können.


26)
tree(a).
tree(b).
tree(c).
%tree(X).
%tree(Y).
%tree(Z).
tree(f(X,Y)) :- tree(X), tree(Y).
tree(g(X)):- tree(X).
tree(h(X,Y,Z)) :- tree(X), tree(Y), tree(Z).