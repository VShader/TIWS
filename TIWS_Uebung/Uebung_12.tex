Uebung 12:

Aufgabe 67:
a)
S->aB->abc
S->Ac->abc

b)
  S		S
  /\	/\
  A c	a B
  |		  |
 ab		  bc
 
c)
G ist Typ 3.

d)
L(G) ist Typ 3 da G Typ 3 und L
durch regulären Ausdruck abc beschrieben werden kann und endlich ist.


Aufgabe 68:
G={V, \sigma, S, P}
V={S, A}
P={
S->\epsilon,	1
S->baaS,		2
S->aA,			3
A->baS,			4
A->abS,			5
}

G ist eine rechtslineare Typ 3 Grammatik.
aabbaa wird erzeugt durch:
S =3> aA =5> aabS =2> aabbaaS =1> aabbaa


Aufgabe 69:
G=(S, \Sigma, S, P)
P={
S->\epsilon,
S->aSB
}

M=({z0, z1}, \Sigma, \Gamma, \Delta, z0, #)
\Delta ={
(z,0,0,\epsilon)
()
}


Aufgabe 71:
a) <var> ::= <var1> +
	 --------->
var-|
     -|var1|-->

b) <var> ::= <var1> ?


Aufgabe 72:
a)
Fakt::='a-z''A-Z'^*|'a-z'* | '(' 

Regel::= Fakt '.' | 