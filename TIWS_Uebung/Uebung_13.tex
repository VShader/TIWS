Uebung 13.

Aufgabe 77:
a)
Zwei Grammatiken sind equivalent wenn alle Wörter der Grammatik G1 in G2 vorkommen
und keine Weiteren.

b)
\Sigma Bin ={0,1} G1=(S, \Sigma Bin, S, P) 
P1={
S->\epsilon,
S->0S01
}

G2=({S, A}, \Sigma Bin, S, P)
P2={
S->\epsilon,
S->0SA,
A->01
}


Aufgabe 78:
a)
PCP mit nur einem Buchstaben
((1,1),(1,11),(11,1))
ja

b)Es muss ein Wortpaar geben das gleichlang ist oder es muss zu einem Wortpaar das ungleich lang ist ein Gegenstück geben bei dem u und v vertauscht sind.

c)
siehe a) Mögliche Lösung sind 1, 2.3, 2.1.3, 1.2.1.3.1 ...


Aufgabe 79:
a) ((011,0101),(01,0),(01,101))
11 011011\neq 01010101
23 0101 = 0101

b) ((01,001),(01,100),(0,11))
Es kann keine Indexfolge geben, da 


Aufgabe 80:
a)
Die Wortpaare lassen sich als struct arrays darstellen.
struct Pair{
char u[];
char v[];
};
Mit 2 Schleifen, eine um die anzahl der Wörter zu erhöhen und eine um die Wörte zuverbinden.
Bsp.: Prüfe 1 Wort
Pair words[];
for(int i=0; i<WordSize; ++i) check(words, i);

Bei 2 Wörter
Pair words[];
for(int i=0; i<WordSize; ++i) check(words, i);

int wcount=1;
while(true)
{
}
b) Es lohnt sich nur Wortpaare zu untersuchen, die Gleichgroß sind.