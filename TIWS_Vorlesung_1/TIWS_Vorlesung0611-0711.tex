\usepackage{ wasysym }

Neben Arithmetik spezielle Notation für Listen!

list(1,list(2,list(3,nil)))
								nicht mischen!!!
[1|[2|[3|[]]]]

Quicksort: Teile und Hersche / Divide and Conquer
Anwendung: Generate & Test-Verfahren

alle Permutationen			ist sortiert
der Eingabe erzeugen

psort([3,2,1],Zs) :- permutation([3,2,1]), ordered([3,2,1]) f
					 permutation([3,1,2]), ordered([3,1,2]) f
					 permutation([2,1,3]), ordered([2,1,3]) f
					 permutation([2,3,1]), ordered([2,3,1]) f
					 permutation([1,2,3]), ordered([1,2,3]) \checked
					 permutation([1,3,2]), ordered([1,3,2]) f
					 
ordered([]).
ordered([X]).
ordered([X,Y|Xs]) :- X<Y, ordered([Y|Xs]).

Schöne Anwendung: n-Damen-Problem!



Kowalski-Idee: Programming = Logic + Control
			   In Prolog verwirklicht?
			   D.h. nur Logik angeben, keine Gedanken zur Auswertung
			   D.h. sind "und" und "oder" kommutativ?


Fakten- und Regelreihenfolge

"oder" kommutativ?

Bsp.:
%istdrin(X,Xs): X ist in Liste Xs enthalten
istdrin(X,[X|Xs]).
istdrin(X,[_|Xs]) :- istdrin(X,Xs).


BILD: Bild_Baum_istdrin1


%istdrin(X,Xs): X ist in Liste Xs enthalten
istdrin(X,[_|Xs]) :- istdrin(X,Xs).
istdrin(X,[X|Xs]).

BILD: Bild_Baum_istdrin2		   

Bei Vertauschung von Regel/Klausel wird Baum gespiegelt!!!
Problem: \infty Äste von rechts nach links!
Struktur der Äste ändert sich nicht, d.h.:
alle Äste bleiben erhalten und behalten gleiche Länge
\Rightarrow endliche Bäume bleiben endlich und \infty Bäume bleiben \infty


Prädikatreihenfolge in Queries und rechten Regelseiten "und" Kommutativ?

Hierbei kann die Struktur verändert werden,
d.h. Äste können
- wegfallen
- hinzukommen
- länger werden
- kürzer werden

Bsp.: Folie: 5, Seite: 8


BILD: noch zu machen!!!!











Endlicher Baum wird viel breiter, bleibt aber endlich, Suchraum größer aber es wird alles gefunden!
Schlimmer, wenn \infty Äste reinkommen!


%elter(X,Y): X ist Elternteil von Y
%vorfahre(X,Y): X ist Vorfahre von Y

Transitiver Abschluss von elter
vorfahre(X,Y) :- elter(X,Y).
vorfahre(X,Y) :- elter(X,R),vorfahre(R,Y).

Allgemein: Transitiver Abschluss transvonrel(X,Y) einer Relation rel(X,Y).:
transvonrel(X,Y) :- rel(X,Y).
transvonrel(X,Y) :- rel(X,R), transvonrel(R,Y).

Transitiver Abschluss: mindestens 1 Schritt der Ursprungsrelation.
Transitiver Reflexiver Abschluss: auch kein Schritt der Ursprungsrelation erlaubt.

Bsp.: Weg ist transitiv-reflexiver Abschluss von Kante!
Allgemein: Transitiv-reflexiver Abschluss einer Relation rel(X,Y).
transrefvonrel(X,X).
transrefvonrel(X,Y) :- rel(X,R),transrefvonrel(R,Y).

Bsp.:
weg(X,X).
weg(X,Y) :- kante(X,R), weg(R,Y).


Explizite Backtrackingkontrolle mittels der Atomformel Cut.

Nullstellige Relationen werden ohne Klammer geschrieben.
Bsp.: Folie: 5, Seite: 9



BILD: Bild_Baum_Folie_5.9



Semantik des Cuts '!'
- tritt nur auf rechten Regelseiten auf
- ist erfolgreich
- schneidet beim Backtracken alle Alternativen von den Prädikaten ab, die in seiner Regel links von ihm stehen.



BILD: Bild_Cut_Folie_5.10