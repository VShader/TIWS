Terme: Die einzige Datenstruktur in Prolog
Zusätzlich zu Konstanten und Variablen brauchen wir Funktoren (gleiche Syntax wie Relationen und Konstanten).

Bsp. für Terme \mathbb{N} in symbolischer Notation.
5 \rightarrow s(s(s(s(s(o))))) \rightarrow $s^$5(o)
1-Stelliger Funktor \nearrow \uparrow Konstante

Terme-Definition.:										Bsp.: Listen
Induktionsanfang: - Jede Konstante c ist ein Term		nil
				  - Jede Variable X ist ein Term		2-stelliger Funktor

n \rightarrow n+1: - Für jeden n-stelligen Funktor f 	list
und n Terme t1,...,tn ist f(t1,...,tn) ein Term.

Term \rightarrow Baum
Bsp. Konstanten a, b Variablen X Funktoren $h^$(3), $g^$(1), $f^$(2)
$h^$(3)($g^$(1)($g^$(1)(a),$f^$(2)(X,b),$f^$(2)($f^$(2)(a,$g^$(1)(X)),b))

Gesucht: baum: Terme \rightarrow Baum-Darstellung

Induktionsanfang	baum(c)=c	Konstante c
					baum(X)=X	Variable X
					
n \rightarrow n+1	baum(f(t1,...tn))	$f^(n)$ Funktor
										t1,...,tn Terme
										
Wir suchen nach Substitutionen der Variablen in Query(Z,...), so dass Substitut in Relation.
Def.: (Substitution)
Eine Substitution ist eine endliche Menge (möglicherweise leer) von Paaren der Form Xi,ti, wobei Xi eine Variable ist und ti ein Term ist.
Pro Substitution höchstens eine Substitution pro Variable geschrieben:[Z1/t1,...,Zn/tn]

Bsp.: [Z1/sarah,Z2/isaak]

Def.:(Anwendung einer Substitution und Instanz)
Seien t ein Term und Sub eine Substitution.
Das Ersetzten der Variablen durch die rechten Seiten der Substitution innerhalb von t heißt Anwendung von Sub auf t und wird geschrieben als tsub.
tsub heißt spezieller als t oder Instanz von t.
umgekehrt t allgemeiner als tsub.

Bsp.: t=$s^5$(Y) sub=[Y/$s^7$(x)]
		tsub=$s^5$($s^7$(x))=$s^12$(x)
		-t=list(X,list(b,list(Y,Z)))
		tsub=list(a,list(b,list(c,nil)))
		sub=[X/a,Y/c,Z/nil]
		
Regeln: (braucht man meistens Induktionsvorschrift)
Neben Fakten kann ein Prolog Programme auch Regeln enthalten.