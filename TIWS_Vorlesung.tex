\documentclass[11pt]{scrartcl}
\begin{document}

Einführung:
Ziele des Studiums:
Probleme lösen!
Projekte durchführen!
Abstraktionsfähigkeit!!!
Prozesse automatisieren!
Teamarbeit!
Komplexe Systeme entwickeln!

Ordnung: \leq \subsetneq MxM		(5,3) nicht
Menge: M							(3,5) ja

Bsp.: M={1,2,3} herkömmliches <
	  L={(1,2),(2,2),(3,3),(2,1),(3,1),(3,2)} \cup <
	  
	  
	  
1. Wissensbasierte Systeme

Repräsentation			Inferenzmechanismus
jeweils Verschieden		Immer gleich
Pro Anwendungsgebiet
Problembezogen			Problemunabhängig
Mensch					Maschine
Programm				Compiler
Viele					1 mal

2. Einführung Logikprogrammierung

-Aussagenlogik: DTTI, Mathe
-Prädikatenlogik: aus Mathe II

Aussagenlogik:
Induktionsanfang -atomare Formel: "Hildegard hat blonde Haare!" \widehat{=} A Wahr oder Falsch
n->n+1			 -A, B atomare Formel => (A\lorB), (A\landB), \lnotA
\end{document}