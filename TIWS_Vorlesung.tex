\documentclass[11pt]{scrartcl}
\begin{document}

Einführung:
Ziele des Studiums:
Probleme lösen!
Projekte durchführen!
Abstraktionsfähigkeit!!!
Prozesse automatisieren!
Teamarbeit!
Komplexe Systeme entwickeln!

Ordnung: \leq \subsetneq MxM		(5,3) nicht
Menge: M							(3,5) \check{mark}

Bsp.: M={1,2,3} herkömmliches \lt
	  L={(1,2),(2,2),(3,3),(2,1),(3,1),(3,2)} \cup \surd
	  
	  
	  
1. Wissensbasierte Systeme

Repräsentation			Inferenzmechanismus
jeweils Verschieden		Immer gleich
Pro Anwendungsgebiet
Problembezogen			Problemunabhängig
Mensch					Maschine
Programm				Compiler
Viele					1 mal

2. Einführung Logikprogrammierung

-Aussagenlogik: DTTI, Mathe
-Prädikatenlogik: aus Mathe II

Aussagenlogik:
Induktionsanfang -atomare Formel: "Hildegard hat blonde Haare!" \widehat{=} A Wahr oder Falsch
n->n+1			 -A, B atomare Formel => (A\lor B), (A\land B), \lnot A
					   zusammengesetzte		or			and		not
					   						zusammengesetzte Formeln

\begin{tabular}{lcr}
A & B & A \lor B & A \land B & \lnot A \\
0 & 0 & 0 & 0 & 1 \\ 
0 & 1 & 1 & 0 & 1 \\		  0      1				 0		1
1 & 0 & 1 & 0 & 0 \\		((A \lor B) \land (\lnot A \lor B))
1 & 1 & 1 & 1 & 0 \\			1					1
\end{tabular}							1
							A=0, B=1
							Verschachtelungstiefe = 3
							
Ist Aussagenlogik entscheidbar?
Das heißt ist folgendes Problem algotithmisch lösbar:
Eingabe:	Eine beliebige aussagenlogische Formel
Problem:	Gibt es eine Belegung bel.: Atome->{0,1}, so dass bel(F)=1 ?

Problem ist entscheidbar durch Entscheidungsalgorithmus.
Einfach alle 2$^n$ Belegungen austesten.
Falls eine 1 liefert: output ("JA")
			   sonst: output ("NEIN")
			   
Aussagenlogik nicht aussagekräftig genug
Prädikatenlogik 1. Stufe
Problem: Unentscheidbar!!!

Quantoren: \exists , \forall
Variablen: x, g
Funktoren: +, -
Relationen: |x-x0| <delta

	Problem
---------------------------------


Grundlagen:

Prolog basiert auf Hornklausellogik \leq Prädikatenlogik 1. Stufe
Programme bestehen aus
Fakten: groß(erwin).
Regeln: klaus liebt alle, die groß und schlank sind.
		liebt(klaus,x):- groß(x),schlank(x).
_______
Anfrage: ?- liebt(klaus,erwin).


3. Prolog-Grundlagen

-Syntax von Prolog und Queries
-Wissensbasis
-erste Programmiertechniken kennen lernen
induktiv Programmieren

Fakten:
1. Form von Prolog-Statements
Relation vater ist 2-stellige Relation
bisher nur Paar(abraham, isaak)in Relation vater
hier: Relation beschrieben durch Aufzählung der Elemente
Semantik durch Kommentar festlegen:
%vater(X,Y): X ist Vater von Y.
Query: "Ist abraham Vater von isaak?"
		(abraham,isaak)\in vater?
		
Def.: (Relation, Prädikat)
a) Eine n-stellige Relation über Menge M ist eine Teilmenge des n-fachen Kartesischen Produkts
M x...x M = $M^$n über M
  n-mal

Bsp.: a) M = \mathbb{N} = {0,1,2,...}
		 $M^$3 = {(0,1,2),(0,0,0),(1,0,2),...}
		 3-stellige Relation über \mathbb{N}:
mult = {(0,0,0),(0,1,0),(1,0,0),(1,1,1),(1,2,2),(2,1,2),(2,2,4),...}
	   {(x,y,z) \\x,y,z \in \mathbb{N}:Z=x*y}
	   
mult:\mathbb{N}X\mathbb{N}->\mathbb{N}		\mathbb{N}X\mathbb{N}X\mathbb{N}
mult(x,y)->x*y

Man kann jede Funktion als Relation darstellen, indem man ein Argument dazunimmt für den Funktionswert.
Umgedreht geht es nicht, denn z.B.: (abraham,isaak) \in vater
									(abraham,sarah) \in vater
es kann mehrere Werte im rechten Argument zu gleichem Wert davor geben.
		Funktion \rightarrow Relation <-Mächtiger
				 \nleftarrow
		Determinismus \subseteq Nichtdeterminismus
		
		
Def.:
b) Prädikat n-stelliges Prädikat über M ist eine Funktion P:$M^$n->{0,1}
c) n-stelliges Prädikat P zur n-stellien Relation R
   P(a1,...,an)=1 \Leftrightarrow (a1,...,an) \in R
   
 Folie 2: 4 Relationen : vater, mutter 2-stellig
 						 männlich, weiblich 1-stellig
 						 
Syntax von Konstanten und Relationen;
Zeichenfolge, die mit Kleinbuchstaben beginnen und aus Buchstaben, Ziffern und '_' bestehen kann

Bsp.: meier, herr_schulze, u33, a2, vater, sarah, ...

Unterscheidung durch Position;
	außen\rightarrow Relation
	innen\rightarrow Konstante
	Später: Zahlen als Konstanten!
	

Anfragen (Queries):
Fragt: (abraham,isaak) \in vater
Wer sind Kinder von abraham?
Müssten wir alle Kombinationen ?- vater(abraham,...) durchsuchen.
\rotatebox[origin=c]{180}{$\Lsh$} Variablen
Kann anstelle der Konstanten auch Variablen verwenden!
Syntax von Variablen, wie Konstanten, aber mit Großbuchstaben oder '_' beginnen.

Bsp.: X, Ergebnis, _x, _123, _
							\nwarrow anonyme Variable (wird nur 1 mal gebraucht)

Query:?- vater(terach,Z).
Frage: Kinder von terach?
Z=abraham;
Z=nachor;
Z=haran.

Es wird nach einen Ersatz (Substitution) der Variablen in der Query gesucht, mit der man Elemente der Relation erhält.
Wir schreiben statt Z = haran \rightarrow [Z/haran]
?- vater(Z1,Z2)\rightarrow [Z1/terach, Z2/abraham]
									.
									.
									.
							[Z1/haran, Z2/yiscah]
							
Variablen in Queries sind existenz - quantifiziert

Terme: die einzige Datenstruktur von Prolog
Def.: (natürlich Zahl in Symbolischer Darstellung)
		0\in \mathbb{N}:Konstante o
		n\in \mathbb{N} \Rightarrow n+1\in \mathbb{N}: Funktor s 1-stelliger Funktor

Bsp.: 3 = s(s(s(o)))

\end{document}